\documentclass[12 pt, letterpaper,spanish]{article}
\usepackage[spanish]{babel}
\usepackage[utf8]{inputenc}
\usepackage{amssymb}
\usepackage{amsmath}
\usepackage{enumitem}

\newenvironment{exercise}[2][Ejercicio]{\begin{trivlist}
\item[\hskip \labelsep {\bfseries #1}\hskip \labelsep {\bfseries #2.}]}{\end{trivlist}}

\newenvironment{sol}
    {\emph{Solución:}
    }
    
\title{Apuntes de Matemáticas Biológicas.\\ Semestre A20.}
\author{}
\date{June 2020}

\begin{document}

\maketitle

\begin{exercise}{3.9.1, Hillen}
\hfill \break
\textbf{El método de $C^{14}$.}
El método del $C^{14}$ es usado para estimar la edad de objetos arqueológicos.
Es sabido que seres vivos acumulan el isótopo radioactivo $C^14$ durante su tiempo de vida, hasta una concentración $c_0$.
Cuando el organismo muere, el $C^{14}$ radioactivo decaé con una vida media de $T_{1/2} = 5760 \text{ años}$.

Arqueólogos encontraron una pieza de madera en el delta del Nilo que mostraban una concentración del $75 \%$ de $c_0$.
Estime la edad de esta pieza de madera.
¿Pudo Tutankamón estar sentado en un bote hecho del mismo árbol del cuál provino esta pieza de madera?
\end{exercise}
Solución: La ecuación diferencial del decaemiento se puede expresar mediante:
   \begin{center}
        $\frac{dC}{dt} = -\alpha(C)$
    \end{center}
Esta ecuación diferencial se puede resolver por medio de separación de variables
    \begin{center}
        $\frac{dC}{C} = -\alpha(dt)$
    \end{center}
    \begin{center}
        $\int\limits_{Ci}^{Cf} \frac{dC}{C} \cdot = -\alpha(dt)$
    \end{center}
    \begin{center}
        $ln(cf)-ln(ci) = -\alpha(t) $     
    \end{center}
    \begin{center}
        $ln \frac{Cf}{Ci} = -\alpha(t) $     
    \end{center}

De esta forma si Cf=1/2Ci, entonces

    \begin{center}
        $ln \frac{1/2Ci}{Ci} = -\alpha(t) $     
    \end{center}
    
    \begin{center}
        $ln1/2 = -\alpha(t) $     
    \end{center}
    
     \begin{center}
        $\alpha =-\frac{ln 1/2}{t}= 1.20x10^{-4} \text{ años}^{-1}$ 
        , siendo t = 5760 años
    \end{center}
Para saber cuantos años tiene el trozo de madera, se sabe que posee Cf = 0.75 Ci, por lo tanto

    \begin{center}
        $t =-\frac{\frac{1/2Ci}{Ci}}{\alpha }= 2397.35 \text{ años}$ 
    \end{center}

\begin{exercise}{3.9.2 Hillen}
\hfill\break
Psicólogos interesados en aprender acerca de la teoría de aprendizaje.
Una curva de aprendizaje es la gráfica de una función $P(t)$, el desempeño de alguien aprendiendo alguna habilidad nueva como una función del tiempo de entrenamiento $t$.
\begin{enumerate}[label=\alph*]
    \item  ¿Qué representa $dP/dt$?
    
    \item Discuta por qué la ecuación diferencial
    \begin{center}
        $\frac{dP}{dt} = k(M-P)$
    \end{center}
    donde $k$ y $M$ son constantes positivas, es un modelo razonable para el aprendizaje.
    ¿Cuál es el significado de $k$ y de $M$?
    ¿Cuál sería una una condición inicial razonable para el modelo?
    Incluya una gráfica de $dP/dt$ vs $P$ como parte de la discusión.
    
    \item Haga un bosquejo cualitativo de las soluciones de la ecuación diferencial
\end{enumerate}
\end{exercise}
\begin{sol}
\begin{enumerate}[label=\alph*]
    \item La ecuación diferencial $dP/dt$ representa el cambio del desempeño de alguien aprendiendo una habilidad con respecto al tiempo. Esto es, la velocidad de aprendizaje.
    
    \item Con respecto a la ecuación diferencial
    \begin{center}
        $\frac{dP}{dt} = k(M-P)$
    \end{center}
    la constante $k$ es la tasa de decrecimiento, mientras que $M$ es el desempeño límite.
    
    Una condición inicial razonable para el modelo es 
    
    \item 
\end{enumerate}
\end{sol}


\begin{exercise}{3.9.3 Hillen}
\hfill\break
\textbf{Cosecha.}
El modelo (logístico) de Verhulst para crecimiento poblacional es 
\begin{center}
    $\dot{x} = a u \left( 1 - \frac{u}{K} \right)$
\end{center}
donde $K$ denota la \textit{carrying capacity} y $a$ la tasa de reproducción.
Asumimos que la cantidad consechada es proporcional al tamaño de la población, con constante de proporcionalidad $c$.
Entonces el modelo es
\begin{center}
    $\dot{x} = a u \left( 1 - \frac{u}{K} \right) - cu$.
\end{center}
Grafique el campo vectorial y encuentre los estados estacionarios.
¿Para qué valor de $c$ la población se extingue?
Dé una explicación biológica
\end{exercise}
\begin{sol}

\end{sol}



\begin{exercise}{3.9.4 Hillen}
\hfill\break
\textbf{Pesca.}
En este modelo consideraremos tres modelos simples de pesca.
Sea $N(t)$ la población de peces a tiempo $t$.
En ausencia de pesca, se asume que la población crece logísticamente
\begin{center}
    $\dot{N} = r N\left( 1-\frac{N}{K} \right)$
\end{center}
donde $r$ es la tasa de cecimiento intrínseca de la población, y $K>0$ es la carrying capacity de la población de peces.
Los efectos de pesca son modelados con un término adicional en la ecuación para $N$.
Los tres modelos son los siguientes:
\begin{center}
\begin{tabular}{c l}
     \textbf{Modelo 1:} & $\dot{N} = r N\left( 1-\frac{N}{K} \right) - H_1$;  \\
     \textbf{Modelo 2:} & $\dot{N} = r N\left( 1-\frac{N}{K} \right) - H_2 N$;\\
     \textbf{Modelo 3:} & $\dot{N} = r N\left( 1-\frac{N}{K} \right) - H_3 \frac{N}{A+N}$,
\end{tabular}
\end{center}
donde $H_1$, $H_2$,  $H_3$ y $A$ son constantes positivas.
\begin{enumerate}[label=\alph*]
    \item Para cada modelo, de una interpretación biológica del término de pesca.
    ¿En que difieren?
    ¿Cuál es el significado de las constantes $H_1,H_2,H_3$ y $A$?
    
    \item Critique el Modelo 1. 
    ¿Por qué no es biológicamente realista?
    
    \item ¿Cuál de los Modelos 1 y 2 cree que es mejor y por qué?
\end{enumerate}
\end{exercise}

\end{document}
